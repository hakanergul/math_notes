\documentclass[10pt,a4paper]{article}
\usepackage[utf8]{inputenc}
\usepackage[turkish]{babel}
\usepackage{amsmath}
\usepackage{amsfonts}
\usepackage{amssymb}
\usepackage{amsthm}
\author{Hakan ERGÜL}

\theoremstyle{definition} \newtheorem{teo}{Teorem}

\setlength{\parskip}{3pt plus 2pt}
\setlength{\parindent}{10pt}
\setlength{\oddsidemargin}{0.5cm}
\setlength{\evensidemargin}{0.5cm}
\setlength{\marginparsep}{0.75cm}
\setlength{\marginparwidth}{2.5cm}
\setlength{\marginparpush}{1.0cm}
\setlength{\textwidth}{150mm}
\begin{document}

\begin{enumerate}
\item[\textbf{4.}] $U\in\tau$ ve $ A $ kapalı olduğundan $A ^{c}\in \tau$ olarak veriliyor. O halde 
\begin{displaymath}
U\setminus A = U\cap A^{c} \in \tau
\end{displaymath}
olup $ U\setminus A $ açık bir kümedir. 
Ayrıca $ A $ kapalı ve $ U $ açık olduğundan $ U^{c} $ kapalıdır. Bu takdirde
\begin{displaymath}
A\setminus U=A\cap U^{c}
\end{displaymath}
kapalıdır. Çünkü iki kapalı kümenin arakesiti de kapalıdır.

\item[\textbf{8.}] 
\begin{enumerate}
\item[\textbf{a)}]
$ x\in \overline{A\cap B} $ keyfi elemanını alalım. Bu takdirde
\begin{eqnarray}
&\forall &\!\!\!\!\!U\in \mathcal{N}_{x} :A \cap B\cap U\neq\emptyset \nonumber \\
\Rightarrow &\forall &\!\!\!\!\!U\in \mathcal{N}_{x} :[A \cap U\neq\emptyset\, \wedge\, B\cap U\neq\emptyset]\nonumber \\
\Rightarrow &\forall &\!\!\!\!\!U\in \mathcal{N}_{x} :[A \cap U\neq\emptyset]\, \wedge\, \forall U\in \mathcal{N}_{x} :[B \cap U\neq\emptyset]\nonumber
\end{eqnarray}

olup $ x \in \overline{A} $ ve $ x \in \overline{B} $ olur. Yani $ x \in \overline{A}\cap \overline{B}\, $ dir. Sonuç olarak $ x \in \overline{A\cap B} $ iken $ x \in \overline{A}\cap \overline{B} $ bulduk, yani
\begin{displaymath}
\overline{A\cap B}\subseteq \overline{A}\cap \overline{B}
\end{displaymath}
elde edilir.
   \paragraph{}
Diğer yandan $ x \in \overline{A}\cap \overline{B} $ alalım. $ x \in \overline{A} $ ve $ x \in \overline{B}\, $ dir. Yani
\begin{eqnarray}
&\forall & \!\!\!\!\!U\in \mathcal{N}_{x} :[A \cap U\neq\emptyset]\, \wedge\, \forall U\in \mathcal{N}_{x} :[B \cap U\neq\emptyset]\nonumber \\
\Rightarrow &\forall &\!\!\!\!\!U\in \mathcal{N}_{x} :[A \cap U\neq\emptyset\, \wedge \,B \cap U\neq\emptyset] \nonumber
\end{eqnarray}
olur. Burada $ A\cap U\neq\emptyset $ ve $ B\cap U\neq\emptyset $ olması $ A\cap B\cap U\neq\emptyset $ olmasını gerektirmez. Zira $ A\cap B=\emptyset $ iken $ A\cap B\cap U=\emptyset $ olur Bu nedenle
\begin{displaymath}
\forall U \in \mathcal{N}_{x} : (A\cap B)\cap U\neq\emptyset
\end{displaymath}
diyemeyiz. Yani $ x \in \overline{A}\cap \overline{B} $ iken $ x \in \overline{A\cap B} $ olmayabilir. Bu da $ \overline{A}\cap \overline{B}\subseteq \overline{A\cap B} $ olamayabileceğini gösterir.

\item[\textbf{b)}]
\textbf{a)} şıkkından farklı bir yol izleyelim. $ \overline{A} $, $ A $ yı içeren tüm kapalı kümelerin arakesiti olduğundan ve bu arakesitin sonucu kapalı olduğundan "$ \overline{A} $, $ A $ yı içeren en küçük kapalı kümedir" diyebiliriz.
\paragraph{}
Öncelikle
\begin{displaymath}
A_{1}\subseteq \overline{A_{1}},\,\, A_{2}\subseteq\overline{A_{2}},\,\,  ... \Rightarrow \bigcap A_{\alpha}\subseteq \bigcap \overline{A_{\alpha}}
\end{displaymath}
olduğu kolayca görülür. $ \bigcap \overline{A_{\alpha}} $, kapalı keyfi sayıda kümenin arakesiti olup kapalıdır. $ \overline{\bigcap A_{\alpha}} $ nın, $ \bigcap A_{\alpha} $ yı içeren en küçük kapalı küme olduğu göz önüne alınırsa
\begin{displaymath}
\bigcap A_{\alpha} \subseteq \overline{\bigcap A_{\alpha}} \subseteq \bigcap \overline{A_{\alpha}}
\end{displaymath}
olur. Sonuç olarak
\begin{displaymath}
\overline{\bigcap A_{\alpha}} \subseteq \bigcap \overline{A_{\alpha}}
\end{displaymath}
elde ederiz.
\paragraph{}
Tersinin her zaman geçerli olmadığını \textbf{a)} şıkkından biliyoruz.

\item[\textbf{c)}] 
Öncelikle $ \overline{A \setminus B} \subseteq \overline{A} \setminus \overline{B} $ daima sağlanmaz.  $  \mathbb{R} $ standart uzayında $ A=(0, 1) $ ve $ B=(1,2) $ alınırsa 
\begin{equation}
\overline{A\setminus B}= \overline{(0, 1)} = [0, 1]
\end{equation}
ve
\begin{equation}
\overline{A} \setminus \overline{B}=[0, 1]\setminus [1, 2]= [0, 1)
\end{equation}
olup (1) ve (2) den $ \overline{A \setminus B} \nsubseteq \overline{A} \setminus \overline{B} $ olduğu görülür.\\
\\
$  \overline{A} \setminus \overline{B}\subseteq \overline{A \setminus B} $ olduğunu gösterelim. İki durumda inceleyelim.\\

i) $A\setminus B= \emptyset$ durumunda
\begin{eqnarray}
A\setminus B=\emptyset &\Rightarrow & A\subseteq B \nonumber\\
&\Rightarrow & \overline{A}\subseteq \overline{B}  \nonumber
\end{eqnarray}
olur. Buradan
\begin{displaymath}
\overline{A\setminus B}= \overline{\emptyset }=\emptyset =\overline{A}\setminus \overline{B}
\end{displaymath}
olup istediğimiz özellik sağlanır.\\
\\
ii) $A\setminus B\neq \emptyset$ durumunu inceleyelim. $x\in \overline{A}\setminus \overline{B}$ alıyoruz. Yani
\begin{eqnarray}
\forall U\in \mathcal{N}_{x} :[A \cap U\neq\emptyset ] \, \wedge\, \exists U_{0}\in \mathcal{N}_{x} :[B \cap U_{0}=\emptyset]\nonumber 
\end{eqnarray}
dir. $\exists U_{0}\in \mathcal{N}_{x} $ için $B \cap U_{0}=\emptyset $ ise $x\notin B$ diyebiliriz, aksi halde arakesitin sonucu en azından $x$ i içerir ve boştan farklı olurdu. Demek ki $x\in B^{c}$ dir. O halde
\begin{displaymath}
\forall U\in \mathcal{N}_{x}: B^{c}\cap U\neq \emptyset
\end{displaymath}
yazabiliriz. Bu bilgiyi kullanırsak
\begin{eqnarray}
\forall U\in \mathcal{N}_{x} :[A \cap U\neq\emptyset]\, \wedge\, \forall U\in \mathcal{N}_{x} :[B^{c} \cap U\neq\emptyset]\nonumber
\end{eqnarray}
elde ederiz. Çünkü $A\setminus B \neq \emptyset$ ve ayrıca $\forall U\in \mathcal{N}_{x}$ için $A\cap U\neq \emptyset$ ve $B^{c}\cap U\neq \emptyset$ olduğundan 
\begin{eqnarray}
\forall U\in \mathcal{N}_{x} :[A \cap B^{c}\cap U\neq\emptyset] \nonumber
\end{eqnarray}
olur. Dolayısıyla $x\in \overline{A\setminus B}$ bulduk. Sonuç olarak $\overline{A}\setminus \overline{B} \subseteq \overline{A\setminus B}$ bulunur.
\end{enumerate}


\item[\textbf{12.}]
$ (X, \tau) $ bir Hausdorff uzay olsun. $ A\subseteq X $ kümesi üzerindeki alt uzay topolojisinin
\begin{displaymath}
\tau_{A}:= \{A\cap U \mid U \in \tau\}
\end{displaymath}
olarak tanımlı olduğunu biliyoruz. $ x\neq y $ olmak üzere $ x, y \in A $ noktaları alalım. $ (X, \tau) $ Hausdorff uzay olduğundan $ x\in U_{x} $ ve $ y\in U_{y} $ olacak şekilde $ U_{x}, U_{y}\in \tau $ açık kümeleri vardır öyle  ki $ U_{x}\cap U_{y}=\emptyset $ tur. O halde $ A\cap U_{x}, A\cap U_{y} \in \tau_{A} $ olup $ x \in A\cap U_{x} $ ve $ y \in A\cap U_{y} $ dir. Ayrıca
\begin{eqnarray}
(A\cap U_{x})\cap(A\cap U_{y}) &=& A\cap \underbrace{U_{x}\cap U_{y}}_{\emptyset}\nonumber\\
&=&A\cap \emptyset \nonumber\\
&=&\emptyset \nonumber
\end{eqnarray}
elde ederiz. $ x, y \in A $ için $ A\cap U_{x}, A\cap U_{y} \in \tau_{A} $ bulunabildiği için $ (A, \tau_{A}) $ Hausdorff uzaydır.

\item[\textbf{16.}]
$ T_{1}=\mathbb{R} $ standart uzayı, \\$ T_{2}=\mathbb{R}_{K}$ K-topoloji , \\$ T_{3}=\tau_{c} $ sonlu tümleyen topoloji, \\$ T_{4}=\mathbb{R}_{\textit{l}} $ üst limit topolojisi, \\$ T_{5}= $ baz elemanları $(-\infty, a)  $ olan topoloji
\begin{enumerate}
\item[\textbf{a)}]
$K:=\{ 1/n\; |\; n\in \mathbb{Z}^{+}\}$ kümesi için
\begin{displaymath}
\overline{K}=K\cup K'
\end{displaymath}
olduğunu biliyoruz. Bu bilgiden yararlanarak kapanış kümelerini kolayca bulabiliriz.\\
$T_{1}$ topolojisinde $K'=\{0\}$ olup $\overline{K}=K\cup \{0\}$ dır.\\
$T_{2}$ topolojisinde $K'=\emptyset$ olup $\overline{K}=K$ dır.\\
$T_{3}$ topolojisinde $K'=\mathbb{R}$ olup $\overline{K}=\mathbb{R}$ dır.\\
$T_{4}$ topolojisinde $K'=\emptyset$ olup $\overline{K}=K$ dır.\\
$T_{5}$ topolojisinde $K'=\mathbb{R^{+}}$ olup $\overline{K}=\mathbb{R^{+}}$ dır.\\


\item[\textbf{b)}]\begin{teo}
$(X,\tau)$ ve $(X, \tau ')$ topolojileri arasında $\tau \subseteq \tau '$ ilişkisi varsa $\tau$ Hausdorff ise $\tau '$ de Hausdorff'tur.
\end{teo}

\begin{proof}
$x\neq y$ için $U_{x}, U_{y}\in \tau$ vardır öyle ki $x\in U_{x}$, $y\in U_{y}$ ve $U_{x}\cap U_{y}=\emptyset$ dır.$\tau \subseteq \tau '$ olduğundan
\begin{displaymath}
U_{x}, U_{y}\in \tau \Rightarrow U_{x}, U_{y}\in \tau '
\end{displaymath}
dır. Yani $U_{x}$ ve $U_{y}$ de $\tau '$ de açıktır ve Hausdorff şartlarını zaten sağlıyor yukarıdan görüldüğü gibi.
\end{proof}
$T_{1}$ Hausdorff mu? Hausdorff değilse $\mathbf{T_{1}}$ mi?\\
$x\neq y$ için $x\in U_{x}=(a-\varepsilon, a+\varepsilon )$ ve $y\in U_{y}=(b-\varepsilon, b+\varepsilon)$ $\mathbb{R}$ standart uzayında açıklardır. $\varepsilon = \frac{|a-b|}{3}$ alınırsa $U_{x}\cap U_{y}=\emptyset$ olup Hausdorff aksiyomunu sağlar. Dolayısıyla hem Hausdorff hem de $\mathbf{T_{1}}$ uzayıdır.\\

$T_{2}$ Hausdorff mu? Hausdorff değilse $\mathbf{T_{1}}$ mi?\\
Teorem 1'den dolayı Hausdorff'tur dolayısıyla $\mathbf{T_{1}}$ dir.\\

$T_{3}=\tau_{c}$ Hausdorff mu? Hausdorff değilse $\mathbf{T_{1}}$ mi?\\
\underline{X \textit{sonsuz} bir küme ise ;}
\\$x\neq y$ için $x\in U_{x}$ ve $y\in U_{y}$ açıkları alalım. Bunlar $U_{x}, U_{y}\in \tau_{c}$ olup $U_{x}^{c}$ ve $U_{y}^{c}$ sonludur. Şimdi kabul edelim ki $U_{x}\cap U_{y}=\emptyset$ olsun. O halde
\begin{displaymath}
\underbrace{U_{x}^{c}}_{sonlu} \cap \underbrace{U_{y}^{c}}_{sonlu}=X
\end{displaymath}

olup sonlu iki kümenin arakesiti de sonludur. Fakat bu X in sonsuz olmasıyla çelişir. Çelişki $U_{x}\cap U_{y}=\emptyset$ kabulünden kaynaklandı. Demek ki Hausdorff şartını sağlayan açıklar bulamayız. Dolayısıyla  $\tau_{c}$ Hausdorff değildir.\\

$x\neq y$ için $y\in X\setminus \{x\}$, $ x\notin X\setminus \{x\} $ ve $x\in X\setminus \{y\}$, $ y\notin X\setminus \{y\} $ olup tümleyenleri sırasıyla $\}x\}$ ve $\{y\}$ olduğundan açık kümelerdir. Açıkça görülür ki $T_{1}$ aksiyomunu sağlarlar. Dolayısıyla  $\tau_{c}$ bir $T_{1}$ uzaydır.\\

\underline{X \textit{sonlu} bir küme ise ;}\\
Zaten X in her alt kümesi sonlu olup tümleyenleri de X te sonludur. Böylece $\tau_{c}= P(X)$ olur. Ayrık topoloji olduğu için $x\neq y$ için $x\in \{x\}$ ve $y\in \{y\}$ alınırsa $\{x\}\cap \{y\}=\emptyset$ olup Hausdorff olma şartı sağlanır dolayısıyla $T_{1}$ uzayıdır.\\


$T_{4}$ Hausdorff mu? Hausdorff değilse $\mathbf{T_{1}}$ mi?\\
Teorem 1'den dolayı Hausdorff'tur dolayısıyla $\mathbf{T_{1}}$ dir.\\


$T_{5}$ Hausdorff mu? Hausdorff değilse $\mathbf{T_{1}}$ mi?\\
$T_{5}$ topolojisi $(-\infty, a)$ şeklindeki açıklardan oluşur. Bu yüzden herhangi iki açıktan biri diğerinin alt kümesidir. \\
$x\neq y$ keyfi elemanları için $U_{x}$ ve $U_{y}$ açıkları alalım. Kabul edelim ki $x\in U_{x}, y\notin U_{x}$ ve $y\in U_{y}, x\notin U_{y}$ şartı sağlansın. $T_{5}$ topolojisinde $U_{x}\subset U_{y}$ veya $U_{y}\subset U_{x}$ olduğundan kabulümüz çelişki arz eder. $\mathbf{T_{1}}$ aksiyomu sağlanmaz, dolayısıyla $T_{5}$ Hausdorff uzay da değildir.
\end{enumerate}

\item[\textbf{20.}]

\begin{enumerate}
\item[\textbf{a)}]
$ A:= \{x\times y \mid y=0 \}$\\
$ \stackrel{\;\circ}{A}=\emptyset $\\
$ \partial A=A$

\item[\textbf{b)}]
$ B:= \{x\times y \mid x\geq 0\; \wedge \; y\neq 0\}$\\
$ \stackrel{\;\circ}{B}=B$\\
$ \partial B=\{0\}\times \mathbb{R} \cup \mathbb{R}^{+} \times \{0\} $

\item[\textbf{c)}]
$ C:= A\cup B$\\
$ \stackrel{\;\circ}{C}=\mathbb{R^{+}}\times \mathbb{R}$\\
$ \partial C=\{0\}\times\mathbb{R}\cup \mathbb{R^{-}}\times \{0\} $

\item[\textbf{d)}]
$ D:= \{x \times y \mid x \in \mathbb{Q} \}$\\
$ \stackrel{\;\circ}{D}= \emptyset$\\
$ \partial D=\mathbb{R}\times \mathbb{R} $

\item[\textbf{e)}]
$ E:= \{x\times y \mid  0 < x^{2}-y^{2}\leqslant 1 \}$\\
$ \stackrel{\;\circ}{E}=\{x\times y \mid  0< {x^{2} - y^{2}}< 1 \}$\\
$ \partial E=\{x\times y \mid  |x|=|y| \; \vee\; x^{2}-y^{2}=1 \} $

\item[\textbf{f)}]
$ F:= \{x\times y \mid x\geq 0\; \wedge \; y\neq 0\}$\\
$ \stackrel{\;\circ}{F}=\{x\times y \mid  x\neq 0\; \wedge\; y<\dfrac{1}{x} \}$\\
$ \partial F=\{x\times y \mid x=0\; \vee\; y=\dfrac{1}{x} \} $

\end{enumerate}


\end{enumerate}
\end{document}