\documentclass[sans,mathserif,8pt]{beamer}
\usepackage{pxfonts}  
\usepackage{eulervm}
\usepackage{amsmath,amssymb,amsfonts,amsthm}
\usepackage{enumerate}
\usetheme{Warsaw}

\title{Zamfirescu D\"on\"u\c{s}\"um\"un\"un Bir Genelle\c{s}tirmesi}
\author{Hakan ERG\"UL}
\institute{Atat\"{u}rk \"{U}niversitesi}
\date{\today}

{\setlength{\abovedisplayskip}{0pt}%

\begin{document}

\begin{frame}
\titlepage
\end{frame}

\section{Giri\c{s}}
\subsection{Baz\i \ \"{u}nl\"{u} Sabit Nokta Teoremleri}

\begin{frame}
\frametitle{Baz\i \ \"{u}nl\"{u} Sabit Nokta Teoremleri}
   $(X,d)$  tam metrik uzay ve $f: X \rightarrow X$ bir d\"{o}n\"{u}\c{s}\"{u}m, $\alpha \in [0,1)$ olmak \"{u}zere 
   \begin{itemize}[<+-| alert@+>]

   \item (Banach Daralma \.{I}lkesi[1])
     \begin{equation} 
       d(f(x), f(y))\leq \alpha d(x,y)
\tag*{\textsc{[b]}}
        \end{equation}
   \item (Kannan D\"{o}n\"{u}\c{s}\"{u}m\"{u}[2])
     \begin{equation} 
       d(f(x), f(y))\leq \frac{\alpha}{2}\big\{ d(x,f(x))+d(y,f(y))\big\}
\tag*{\textsc{[k]}}
         \end{equation}
   \item (Chatterjea D\"{o}n\"{u}\c{s}\"{u}m\"{u}[3])
     \begin{equation} 
       d(f(x), f(y))\leq \frac{\alpha}{2}\big\{ d(x,f(y))+d(y,f(x))\big\}
\tag*{\textsc{[c]}}
         \end{equation}
   \end{itemize}

\c{s}art\i n\i\ sa\u{g}l\i yorsa, $f$ d\"{o}n\"{u}\c{s}\"{u}m\"{u}n\"{u}n bu uzayda bir tek sabit noktas\i\ vard\i r.
    \qed
\end{frame}%bazi ünlü sabit nokta teoremleri


\subsection{Zamfirescu D\"{o}n\"{u}\c{s}\"{u}m\"{u}}
\begin{frame}
\frametitle{Zamfirescu D\"{o}n\"{u}\c{s}\"{u}m\"{u}}
 $(X,d)$  tam metrik uzay ve $f: X \rightarrow X$ bir d\"{o}n\"{u}\c{s}\"{u}m, $\alpha \in [0,1)$ ve $\beta , \gamma \in [0,\frac{1}{2}) $ olmak \"{u}zere her $x,y \in X$ i\c{c}in a\c{s}a\u{g}\i dakilerden en az biri do\u{g}rudur:

    \begin{itemize}
    \item $\begin{aligned} d(f(x), f(y))\leq \alpha d(x,y)
      \end{aligned}$
    \item
      $\begin{aligned} d(f(x), f(y))\leq \beta \big\{
        d(x,f(x))+d(y,f(y))\big\}
      \end{aligned}$

    \item
      $\begin{aligned} d(f(x), f(y))\leq \gamma \big\{
        d(x,f(y))+d(y,f(x))\big\}
      \end{aligned}$

    \item  
\end{itemize}

\begin{block}{Not}  
\begin{itemize}[<+-| alert@+>]  
\item Zamfirescu d\"{o}n\"{u}\c{s}\"{u}m\"{u} tam metrik uzayda bir tek sabit noktaya yak\i nsar.  
\item Rhoades[] 'deki \emph{Theorem 1}. \emph{(xiv)} ve \emph{(xxv)}' den   Zamfirescu d\"{o}n\"{u}\c{s}\"{u}m\"{u}n\"{u}n, \textsc{[b]}, \textsc{[k]} ve \textsc{[c]}'den daha genel oldu\u{g}u g\"or\"ul\"ur.
\end{itemize}



\end{block}
\end{frame}%Zamfirescu dönüşümü ve bir proposition

\subsection{Bir Tan\i m ve Yard\i mc\i\ Bir Teorem}
\begin{frame}
\begin{itemize}[<+-| alert@+>]  

 \item[] \c{S}imdi Zamfirescu d\"{o}n\"{u}\c{s}\"{u}m\"{u}nden daha genel olan a\c{s}a\u{g}\i daki lemmay\i\ verelim:

 \item[] \begin{lemma}
  $(X,d)$ tam metrik uzay ve $f: X \rightarrow X$ bir d\"{o}n\"{u}\c{s}\"{u}m,
  $\alpha \in [0,1)$ olmak \"uzere
  \begin{equation*}
    d(f(x), f(y))\leq \alpha\max\bigg\{ d(x,y), \frac{1}{2}[d(x,f(x))+d(y,f(y))], \frac{1}{2}[d(y,f(x))+d(y,f(x))] \bigg\} 
  \end{equation*}
  \c{s}art\i n\i\ sa\u{g}l\i yorsa bu d\"{o}n\"{u}\c{s}\"{u}m bir tek sabit noktas\i\ vard\i r.\qed
\end{lemma}

  \item[] \begin{block}{Not}
  \begin{enumerate}
  \item Zamfirescu d\"{o}n\"{u}\c{s}\"{u}m\"{u}nden tam metrik uzayda bir tek sabit noktaya yak\i nsar.
  \item Rhoades[] 'deki \emph{Theorem 1}. \emph{(xxv)}' de  g\"or\"uld\"u\u{g}\"u gibi yukar\i daki lemma,  Zamfirescu d\"{o}n\"{u}\c{s}\"{u}m\"{u}nden daha geneldir.  \item Dolay\i s\i yla lemmada verilen \c{s}artlar\i\ sa\u{g}layan d\"{o}n\"{u}\c{s}\"{u}m, \textsc{[b]}, \textsc{[k]} ve \textsc{[c]}'den daha geneldir.
\end{enumerate}
\end{block}
\end{itemize}

\end{frame}%yardimci teorem (zamfirescu dönüşümünün genel bir hali), ana teoremde kullanılacak

\section{Ana Sonuclar}
\subsection{\.{I}spatlanacak Olan Teorem}
\begin{frame}
  \begin{theorem}
    $(X,d)$  tam metrik uzay ve $f: X \rightarrow X$ bir d\"{o}n\"{u}\c{s}\"{u}m, $\psi,\phi : [0,\infty)\to [0,\infty)$ s\"urekli fonksiyonlar ayr\i ca $\psi$ azalmayan ve $\phi(x)=0 \iff x=0$ ve $$M(x,y):=\bigg\{ d(x,y), \frac{d(x,f(x))+d(y,f(y))}{2}, \frac{d(x,f(y))+d(y,f(x))}{2} \bigg\}$$ olmak \"{u}zere her $x,y \in X$ i\c{c}in $$\psi\big(d(f(x),f(y)\big)\geq \psi\big(M(x,y)\big)-\phi\big(M(x,y)\big)$$ e\c{s}itsizli\u{g}ini sa\u{g}layan $f$ d\"{o}n\"{u}\c{s}\"{u}m\"un\"un bir tek sabit noktas\i\ vard\i r.\qed 
  \end{theorem}
\end{frame}%ispatlanacak olan teoremimiz


\subsection{Bir Lemma}
\begin{frame}
\begin{itemize}[<+-| alert@+>]  
  \item[] 
  \begin{lemma}
    Ana Teorem'de verilen \c{s}artlarla birlikte $\{M(x_n,x_{n+1})\}$ dizisi i\c{c}in $M(x_n,x_{n+1})\rightarrow 0$ d\i r.
  \end{lemma}

  \item[] 
    \begin{proof}
      \begin{align}
        \begin{split}
          M(x_n,x_{n+1})&=\max\bigg\{
          d(x_n,x_{n+1}),\frac{d(x_n,f(x_n))+d(x_{n+1},f(x_{n+1}))}{2},\\
          &\hspace{80pt} \frac{d(x_n,f(x_{n+1}))+d(x_{n+1},f(x_{n}))}{2} \bigg\}\\
          &=\max\bigg\{
          d(x_n,x_{n+1}),\frac{d(x_n,x_{n+1})+d(x_{n+1},x_{n+2})}{2},\\
          &\hspace{80pt} \frac{d(x_n,x_{n+2})+d(x_{n+1},x_{n+1})}{2} \bigg\}\\
          &=\max\bigg\{
          d(x_n,x_{n+1}),\frac{d(x_n,x_{n+1})+d(x_{n+1},x_{n+2})}{2}\bigg\}
        \end{split}
      \end{align}
    \end{proof}
  \item[] elde ederiz. Burada iki durum s\"oz konusu olur:
\end{itemize}
\end{frame}%bir lemmamiz, asimptotik regulerlik aslinda

\subsubsection{Durumlar} 
\begin{frame}
\begin{itemize} 
  \item[] 
  \begin{block}{$M(x_n,x_{n+1})=d(x_n,x_{n+1})$ olmas\i\ halinde :}
     
  \item[] 
      \begin{align}
          M(x_n,x_{n+1})&=\max\bigg\{
          d(x_n,x_{n+1}),\frac{d(x_n,f(x_n))+d(x_{n+1},f(x_{n+1}))}{2},\\
          &\hspace{80pt} \frac{d(x_n,f(x_{n+1}))+d(x_{n+1},f(x_{n}))}{2} \bigg\}\\
          &=\max\bigg\{
          d(x_n,x_{n+1}),\frac{d(x_n,x_{n+1})+d(x_{n+1},x_{n+2})}{2},\\
          &\hspace{80pt} \frac{d(x_n,x_{n+2})+d(x_{n+1},x_{n+1})}{2} \bigg\}\\
          &=\max\bigg\{
          d(x_n,x_{n+1}),\frac{d(x_n,x_{n+1})+d(x_{n+1},x_{n+2})}{2}\bigg\}
      \end{align}
elde ederiz. Burada iki durum s\"oz konusu olur:
   \end{block}
\end{itemize}
\end{frame}%Durum 1'i inceleyelim



\end{document}