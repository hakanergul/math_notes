\documentclass[8pt]{beamer}
\usepackage{pxfonts}  
\usepackage{eulervm}
\usepackage{amsmath,amssymb,amsfonts,amsthm}
\usepackage{enumerate}
\usetheme{Warsaw}

\title{Zamfirescu D\"on\"u\c{s}\"um\"un\"un Bir Genelle\c{s}tirmesi}
\author{Hakan ERG\"UL}
\institute{Atat\"{u}rk \"{U}niversitesi}
\date{\today}

{\setlength{\abovedisplayskip}{0pt}%

\begin{document}

\begin{frame}
\titlepage
\end{frame}

\section{Giri\c{s}}
\subsection{Baz\i \ \"{u}nl\"{u} Sabit Nokta Teoremleri}

\begin{frame}
\frametitle{Baz\i \ \"{u}nl\"{u} Sabit Nokta Teoremleri}
   $(X,d)$  tam metrik uzay ve $f: X \rightarrow X$ bir d\"{o}n\"{u}\c{s}\"{u}m, $\alpha \in [0,1)$ olmak \"{u}zere 
   \begin{itemize}[<+-| alert@+>]

   \item (Banach Daralma \.{I}lkesi[1])
     \begin{equation} 
       d(f(x), f(y))\leq \alpha d(x,y)
\tag*{\textsc{[b]}}
        \end{equation}
   \item (Kannan D\"{o}n\"{u}\c{s}\"{u}m\"{u}[2])
     \begin{equation} 
       d(f(x), f(y))\leq \frac{\alpha}{2}\big\{ d(x,f(x))+d(y,f(y))\big\}
\tag*{\textsc{[k]}}
         \end{equation}
   \item (Chatterjea D\"{o}n\"{u}\c{s}\"{u}m\"{u}[3])
     \begin{equation} 
       d(f(x), f(y))\leq \frac{\alpha}{2}\big\{ d(x,f(y))+d(y,f(x))\big\}
\tag*{\textsc{[c]}}
         \end{equation}
\item[] \c{s}art\i n\i\ sa\u{g}l\i yorsa, $f$ d\"{o}n\"{u}\c{s}\"{u}m\"{u}n\"{u}n bu uzayda bir tek sabit noktas\i\ vard\i r.

   \end{itemize}

    \qed
\end{frame}%bazi ünlü sabit nokta teoremleri


\subsection{Zamfirescu D\"{o}n\"{u}\c{s}\"{u}m\"{u}}
\begin{frame}
\frametitle{Zamfirescu D\"{o}n\"{u}\c{s}\"{u}m\"{u}}
 $(X,d)$  tam metrik uzay ve $f: X \rightarrow X$ bir d\"{o}n\"{u}\c{s}\"{u}m, $\alpha \in [0,1)$ ve $\beta , \gamma \in [0,\frac{1}{2}) $ olmak \"{u}zere her $x,y \in X$ i\c{c}in a\c{s}a\u{g}\i dakilerden en az biri do\u{g}rudur:

    \begin{itemize}
    \item $\begin{aligned} d(f(x), f(y))\leq \alpha d(x,y)
      \end{aligned}$
    \item
      $\begin{aligned} d(f(x), f(y))\leq \beta \big\{
        d(x,f(x))+d(y,f(y))\big\}
      \end{aligned}$

    \item
      $\begin{aligned} d(f(x), f(y))\leq \gamma \big\{
        d(x,f(y))+d(y,f(x))\big\}
      \end{aligned}$

    \item  
\end{itemize}

\begin{block}{Not}  
\begin{itemize}[<+-| alert@+>]  
\item Zamfirescu d\"{o}n\"{u}\c{s}\"{u}m\"{u} tam metrik uzayda bir tek sabit noktaya yak\i nsar.  
\item Rhoades[] 'deki \emph{Theorem 1}. \emph{(xiv)} ve \emph{(xxv)}' den   Zamfirescu d\"{o}n\"{u}\c{s}\"{u}m\"{u}n\"{u}n, \textsc{[b]}, \textsc{[k]} ve \textsc{[c]}'den daha genel oldu\u{g}u g\"or\"ul\"ur.
\end{itemize}

\end{block}
\end{frame}%Zamfirescu dönüşümü ve bir proposition



\subsection{Bir Tan\i m ve Yard\i mc\i\ Bir Teorem}
\begin{frame}
\begin{itemize}[<+-| alert@+>]  

 \item[] \c{S}imdi Zamfirescu d\"{o}n\"{u}\c{s}\"{u}m\"{u}nden daha genel olan a\c{s}a\u{g}\i daki lemmay\i\ verelim:

 \item[] \begin{lemma}
  $(X,d)$ tam metrik uzay ve $f: X \rightarrow X$ bir d\"{o}n\"{u}\c{s}\"{u}m,
  $\alpha \in [0,1)$ olmak \"uzere
  \begin{equation*}
    d(f(x), f(y))\leq \alpha\max\bigg\{ d(x,y), \frac{1}{2}[d(x,f(x))+d(y,f(y))], \frac{1}{2}[d(y,f(x))+d(y,f(x))] \bigg\} 
  \end{equation*}
  \c{s}art\i n\i\ sa\u{g}l\i yorsa bu d\"{o}n\"{u}\c{s}\"{u}m bir tek sabit noktas\i\ vard\i r.\qed
\end{lemma}

  \item[] \begin{block}{Not}
  \begin{enumerate}
  \item Zamfirescu d\"{o}n\"{u}\c{s}\"{u}m\"{u}nden tam metrik uzayda bir tek sabit noktaya yak\i nsar.
  \item Rhoades[] 'deki \emph{Theorem 1}. \emph{(xxv)}' de  g\"or\"uld\"u\u{g}\"u gibi yukar\i daki lemma,  Zamfirescu d\"{o}n\"{u}\c{s}\"{u}m\"{u}nden daha geneldir.  \item Dolay\i s\i yla lemmada verilen \c{s}artlar\i\ sa\u{g}layan d\"{o}n\"{u}\c{s}\"{u}m, \textsc{[b]}, \textsc{[k]} ve \textsc{[c]}'den daha geneldir.
\end{enumerate}
\end{block}
\end{itemize}

\end{frame}%yardimci teorem (zamfirescu dönüşümünün genel bir hali), ana teoremde kullanılacak

\section{Ana Sonuclar}
\subsection{\.{I}spatlanacak Olan Teorem}
\begin{frame}
  \begin{theorem}
    $(X,d)$  tam metrik uzay ve $f: X \rightarrow X$ bir d\"{o}n\"{u}\c{s}\"{u}m, $\psi,\phi : [0,\infty)\to [0,\infty)$ s\"urekli fonksiyonlar ayr\i ca $\psi$ azalmayan ve $\phi(x)=0 \iff x=0$ ve
    \begin{align}
M(x,y):=\bigg\{ d(x,y), \frac{d(x,f(x))+d(y,f(y))}{2}, \frac{d(x,f(y))+d(y,f(x))}{2} \bigg\}
    \end{align}
olmak \"{u}zere her $x,y \in X$ i\c{c}in
\begin{align}
\psi\big(d(f(x),f(y)\big)\leq \psi\big(M(x,y)\big)-\phi\big(M(x,y)\big)
\end{align}
e\c{s}itsizli\u{g}ini sa\u{g}layan $f$ d\"{o}n\"{u}\c{s}\"{u}m\"un\"un bir tek sabit noktas\i\ vard\i r.\qed 
  \end{theorem}
\end{frame}%ispatlanacak olan teoremimiz


\subsection{Bir Lemma}
\begin{frame}
\begin{itemize}[<+-| alert@+>]  

  \item[] 
  \begin{lemma}
    Ana Teorem'de verilen \c{s}artlarla birlikte $\{M(x_n,x_{n+1})\}$ dizisi i\c{c}in $M(x_n,x_{n+1})\rightarrow 0$ d\i r.
  \end{lemma}

  \item[] 
    \begin{proof}
      \begin{align*}
          M(x_n,x_{n+1})&=\max\bigg\{
          d(x_n,x_{n+1}),\frac{d(x_n,f(x_n))+d(x_{n+1},f(x_{n+1}))}{2},\\
          &\hspace{80pt} \frac{d(x_n,f(x_{n+1}))+d(x_{n+1},f(x_{n}))}{2} \bigg\}\\
          &=\max\bigg\{
          d(x_n,x_{n+1}),\frac{d(x_n,x_{n+1})+d(x_{n+1},x_{n+2})}{2},\\
          &\hspace{80pt} \frac{d(x_n,x_{n+2})+d(x_{n+1},x_{n+1})}{2} \bigg\}\\
          &=\max\bigg\{
          d(x_n,x_{n+1}),\frac{d(x_n,x_{n+1})+d(x_{n+1},x_{n+2})}{2}\bigg\}
      \end{align*}
    \end{proof}

  \item[] elde ederiz. Burada iki durum s\"oz konusu olur:
\end{itemize}
\end{frame}%bir lemmamiz, asimptotik regulerlik aslinda

\subsubsection{Durum 1}
\begin{frame}
  \begin{block}{Durum 1. $M(x_n,x_{n+1})=d(x_n,x_{n+1})$ olmas\i\ halinde :}
      \begin{align*}
          \psi\big(d(x_{n+1},x_{n+2})\big)&=\psi \big(d(f(x_{n}),f(x_{n+1}))\big)\\
          &\leq \psi\big(d(x_{n},x_{n+1})\big)- \phi\big(d(x_{n},x_{n+1})\big)\\
          &\leq \psi\big(d(x_{n},x_{n+1})\big)
      \end{align*}
olup $\psi$ azalmayan oldu\u{g}undan $$d(x_{n+1},x_{n+2})\leq d(x_n,x_{n+1})$$ d\i r. Yani $\{d(x_n,x_{n+1})\}$ dizisi azaland\i r.
   \end{block}
\end{frame}%Durum 1'i inceleyelim


\subsubsection{Durum 2} 
\begin{frame} 
  \begin{block}{Durum 2.  $M(x_n,x_{n+1})=\frac{1}{2}\big(d(x_n,x_{n+1})+d(x_{n+1},x_{n+2})\big)$ olmas\i\ hali :}
O halde 
      \begin{align}
        d(x_n,x_{n+1})&\leq \frac{1}{2}\big(d(x_n,x_{n+1})+d(x_{n+1},x_{n+2})\big)\\
                    &\leq d(x_{n+1},x_{n+2}) 
      \end{align}
olur. Ayr\i ca 
      \begin{align}
          \psi\big(d(x_{n+1},x_{n+2})\big) &=\psi \big(d(f(x_{n}),f(x_{n+1}))\big) \\
          &\leq \psi\bigg(\frac{d(x_n,x_{n+1})+d(x_{n+1},x_{n+2})}{2}\bigg) \\
          & \hspace{10pt}- \phi\bigg(\frac{d(x_n,x_{n+1})+d(x_{n+1},x_{n+2})}{2}\bigg) \\
          &\leq \psi\big(d(x_{n},x_{n+1})\big)
      \end{align}
olup $\psi $ azalmayan oldu\u{g}undan $$d(x_{n+1},x_{n+2})\leq d(x_n,x_{n+1})$$ d\i r. Bu ise (2) e\c{s}itsizli\u{g}i ile \c{c}eli\c{s}ir.
   \end{block}
\end{frame}%Durum 2'i inceleyelim


\subsubsection{ \.{I}spat\i n Sonucu :} 
\begin{frame} 
  \begin{block}{Lemman\i n Sonucu:}
Sonu\c{c} olarak  $M(x_{n},x_{n+1})= d(x_n,x_{n+1})$ d\i r. O halde $\{d(x_n,x_{n+1})\}$ dizisi monoton azalan(artmayan) bir dizidir ve bir $r\geq 0$ say\i s\i na yak\i nsar. Bu bilgiyi kullan\i rsak $$\psi\big(d(x_{n+1},x_{n+2})\big)\leq \psi\big(d(x_{n},x_{n+1})\big)-\phi\big(d(x_{n},x_{n+1})\big)$$ olup $n\to \infty$ iken limitini al\i rsak $$\psi(r)\leq \psi(r)-\phi(r)$$ elde ederiz ki bu $r=0$ olmas\i n\i\ gerektirir. Sonu\c{c} olarak $\{d(x_n,x_{n+1})\}\to 0$. \qed
   \end{block}
\end{frame}%lemmanın sonucu



\subsection{Ana Teoremin \.{I}spat\i }
\begin{frame}
  \begin{block}{Ana Teoremin \.{I}spat\i }
    \"{O}ncelikle $\{x_n\}$ dizisinin Cauchy dizisi oldu\u{g}unu g\"osterelim. Varsayal\i m ki $\{x_n\}$ bir Cauchy dizisi olmas\i n. O halde $\{x_n\}$ dizisinin \"oyle iki $\{x_{n(k)}\}$ ve $\{x_{m(k)}\}$ altdizisi vard\i r ki $\varepsilon_0 > 0$ olmak \"uzere $m(k)>n(k)\geq k$ olacak \c{s}ekildeki her $k\in \mathbb{N}$ i\c{c}in $$d(x_{m(k)},x_{n(k)})\geq \varepsilon_0$$ d\i r. Dahas\i\ yukar\i daki \c{s}artlar\i\ sa\u{g}layan $m(k)$ say\i s\i n\i\ $n(k)$ dan b\"uy\"uk en k\"u\c{c}\"uk say\i\ olarak se\c{c}ersek $$d(x_{m(k)-1},x_{n(k)})<\varepsilon_0$$ elde ederiz. Buradan
    \begin{align}
      \varepsilon_0 &\leq d(x_{m(k)},x_{n(k)})\\
      &\leq d(x_{m(k)},x_{m(k)-1})+d(x_{m(k)-1},x_{n(k)})\\
      & <d(x_{m(k)},x_{m(k)-1})+\varepsilon_0
    \end{align}
olup $k\to \infty$ iken
    \begin{equation}
      d(x_{m(k)},x_{n(k)})\to \varepsilon_0
    \end{equation}
olur.
\end{block}
\end{frame}

\begin{frame}
 \begin{block}{Ana Teoremin \.{I}spat\i }
Yine
    \begin{align}
      d(x_{m(k)},x_{n(k)})&\leq  d(x_{m(k)},x_{m(k)-1})+d(x_{m(k)-1},x_{n(k)-1})+d(x_{n(k)-1},x_{n(k)})\\ 
      &\leq d(x_{m(k)},x_{m(k)-1})+d(x_{m(k)-1},x_{n(k)})+d(x_{n(k)},x_{n(k)-1})\\
      &\hspace{100pt} +d(x_{n(k)-1},x_{n(k)})
    \end{align}
e\c{s}itsizli\u{g}inde $k\to \infty$ iken limit al\i rsak 
     \begin{equation}
        d(x_{m(k)-1},x_{n(k)-1})\to \varepsilon_0
     \end{equation}
elde ederiz.

\end{block}
\end{frame}%ana teoremin ispatı


\begin{frame}
 \begin{block}{Ana Teoremin \.{I}spat\i }
Benzer olarak
    \begin{align}
      d(x_{m(k)},x_{n(k)})&\leq  d(x_{m(k)},x_{m(k)-1})+d(x_{m(k)-1},x_{n(k)})\\ 
      &\leq d(x_{m(k)},x_{m(k)-1})+d(x_{m(k)-1},x_{n(k)-1})+d(x_{n(k)-1},x_{n(k)})
    \end{align}
e\c{s}itsizli\u{g}inde $k\to \infty$ iken limit al\i rsak 
     \begin{equation}
        d(x_{m(k)-1},x_{n(k)})\to \varepsilon_0
     \end{equation}
buluruz.
\end{block}
\end{frame}%ana teoremin ispatı


\begin{frame}
 \begin{block}{Ana Teoremin \.{I}spat\i }
Ayr\i ca
    \begin{align}
      d(x_{m(k)},x_{n(k)})&\leq  d(x_{m(k)},x_{n(k)-1})+d(x_{n(k)-1},x_{n(k)})\\ 
      &\leq d(x_{m(k)},x_{m(k)-1})+d(x_{m(k)-1},x_{n(k)-1})+d(x_{n(k)-1},x_{n(k)})
    \end{align}
e\c{s}itsizli\u{g}inde $k\to \infty$ iken limit al\i rsak 
     \begin{equation}
        d(x_{m(k)},x_{n(k)-1})\to \varepsilon_0
     \end{equation}
elde ederiz.

\end{block}
\end{frame}%ana teoremin ispatı


\begin{frame}
 \begin{block}{Ana Teoremin \.{I}spat\i }
Sonu\c{c} olarak 
Di\u{g}er yandan
    \begin{align}
      \psi(\varepsilon_0)&\leq \psi\big(d(x_{m(k)},x_{n(k)})\big)\\ 
      &= \psi\big(d(fx_{m(k)-1},fx_{n(k)-1})\big)\\
      &\leq \psi\big(M(x_{m(k)-1},x_{n(k)-1})\big)-\phi\big(M(x_{m(k)-1},x_{n(k)-1})\big)    \end{align}
d\i r. Burada (12), (16), (19) ve (22) deki bilgileri kullan\i rsak
     \begin{align}
        \lim_{k\to \infty}M(x_{m(k)-1},x_{n(k)-1})&=\lim_{k\to \infty}\max\bigg\{ d(x_{m(k)-1},x_{n(k)-1}),\\
       &\hspace{10pt}\frac{1}{2}[d(x_{m(k)-1},x_{m(k)})+d(x_{n(k)-1},x_{n(k)})],\\
       &\hspace{10pt}\frac{1}{2}[d(x_{m(k)-1},x_{n(k)})+d(x_{m(k)},x_{n(k)-1})] \bigg\}\\
       &=\max\{\varepsilon_0, 0, \varepsilon_0\}\\
       &=\varepsilon_0
     \end{align}
elde ederiz. 
\end{block}
\end{frame}%ana teoremin ispatı


\begin{frame}
 \begin{block}{Ana Teoremin \.{I}spat\i }
Bunu (25)'te yerine yazarsak
    \begin{align}
      \psi(\varepsilon_0)\leq\psi(\varepsilon_0)-\phi(\varepsilon_0)
    \end{align}
olup  $\phi(\varepsilon_0)\leq 0$ olur ki bu da $\varepsilon_0=0$ olmas\i n\i\ gerektirir. Bu ise ba\c{s}taki kabul\"um\"uzle \c{c}eli\c{s}ir. Demekki kabul\"um\"uz yanl\i \c{s}t\i r. Yani $\{x_n\}$ bir Cauchy dizisidir. $(X,d)$ tam metrik uzay oldu\u{g}u i\c{c}in $\{x_n\}$ bu uzayda bir $x^*$ noktas\i na yak\i nsar. 
Buradan hareketle
    \begin{align}
      \lim_{n\to \infty}M(x_n, x^*)&=\lim_{n\to \infty}\max\bigg\{ d(x_n,x^*),\frac{1}{2}[ d(x_n,x_{n+1})+ d(x^*,fx^*)],\\
      &\hspace{90pt}\frac{1}{2}[ d(x_n,fx^*)+ d(x^*,x_{n+1})]\bigg\}\\
      & =\frac{d(x^*,fx^*)}{2}
    \end{align}
elde ederiz. Bu bilgiyi (2) de yerine yazarsak $n\to \infty$
\begin{align}
  \psi(d(x^*,fx^*))\leq \psi\bigg(\frac{d(x^*,fx^*)}{2}\bigg)-\phi\bigg(\frac{d(x^*,fx^*)}{2}\bigg) 
\end{align}
\end{block}
\end{frame}%ana teoremin ispatı

\begin{frame}
  \begin{block}{Ana Teoremin \.{I}spat\i }
    yani bu da
    \begin{align}
      \phi\bigg(\frac{d(x^*,fx^*)}{2}\bigg)  \leq \psi\bigg(\frac{d(x^*,fx^*)}{2}\bigg)-\psi(d(x^*,fx^*))
    \end{align}
e\c{s}ittir. Bu e\c{s}itsizli\u{g}in sa\u{g} taraf\i\ $\psi$ azalmayan oldu\u{g}u i\c{c}in $d(x^*,fx^*)\neq 0$ oldu\u{g}u s\"urece  negatif  olup $\phi$ 'nin negatif olmamas\i\ ile \c{c}eli\c{s}ir. B\"oylece $d(x^*,fx^*)=0$ olup $x^*$ noktas\i\ $f$ nin bir sabit noktas\i d\i r.\\
\vspace{10pt}
\hspace{5pt} \c{S}imdi sabit noktan\i n tek oldu\u{g}unu g\"osterelim. Kabul edelim ki $x'\neq x^*$, $f$ nin iki sabit noktas\i\ olsun. O halde 
   \begin{align}
     \psi(d(x',x^*))&=\psi(d(fx',fx^*))\\
      &\leq \psi(M(x',x^*))-\phi(M(x',x^*))\\
      & \leq \psi(d(x',x^*))-\phi(d(x',x^*))
   \end{align}
olup $\phi(d(x',x^*))\leq 0$ elde edilir ki $d(x',x^*)=0$ olmas\i n\i\ gerektirir. Bu sabit noktan\i n tek oldu\u{g}unu kan\i tlar. \qed
  \end{block}
\end{frame} %ana teoremin ispatini bitiriyoruuuum

\section{Sonu\c{c}}
\begin{frame}
  \begin{block}{Sonu\c{c}}
     Sonu\c{c} k\i sm\i\ buraya
  \end{block}
\end{frame}
\end{document}