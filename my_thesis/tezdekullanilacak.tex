\documentclass[10pt]{article}
\usepackage[margin=0.5in]{geometry}
%\usepackage[turkish]{babel} %Türkçe bölüm isimleri
\usepackage[utf8]{inputenc} %Türkçe karakterler
%\usepackage[T1]{fontenc} %Türkçe heceleme
\usepackage{amsmath, amsthm, amsfonts, enumerate}

\renewcommand{\abstractname}{ÖZET}
\theoremstyle{plain}
\newtheorem{theorem}{Teorem}
\newtheorem{proposition}{Önerme}
\newtheorem{example}{Örnek}
\theoremstyle{definition}
\newtheorem{definition}{Tanım}


\author{Hakan ERGÜL}
\title{Tez Malzemeleri}
\date{\today}

\begin{document}
\maketitle
\section{TANIMLAR}
Kullanabileceğim tanımlar:

\definition [sabit nokta]{$(X,d)$ bir metrik uzay ve $f:X\to X$ bir fonksiyon olmak üzere $f(x)=x$ şeklindeki $x\in X$ noktasına $f$ fonksiyonunun sabit noktası denir.} 

\definition [periyodik nokta] {$(X,d)$ bir metrik uzay ve $f:X\to X$ bir fonksiyon ve $N>1$ bir doğal sayi olmak üzere $f^1(x)=f(x)$ ve $f^{N+1}(x)=f(f^{N}(x))$ şeklinde tanimlansin. $f^N(x)=x$ şeklindeki $x\in X$ noktasına $f$ fonksiyonunun periyodik noktası denir. $N=1$ için bu $x$ noktası sabit noktadır. Ayrica $f^N$ nin sabit noktaları, $f$ nin periyodik noktalarıdır.}

\definition [yaklaşık(approximate) sabit noktası- $\varepsilon$-sabit nokta] {$(X,d)$ bir metrik uzay ve $f:X\to X$ bir dönüşüm olsun. Eğer bir $\varepsilon>0$ için $d(f(x_0),x_0)<\varepsilon$ olacak şekildeki $x_0\in X$ noktasına $\varepsilon$-sabit nokta denir.}

\definition [coincidence point-çakışık nokta] {$(X,d)$ bir metrik uzay ve $f,g:X\to X$ birer dönüşüm olsun. Eğer $x\in X$ için $f(x)=g(x)$ ise $x$ noktası $f$ ve $g$ fonksiyonlarının çakışma noktasıdır denir. Eğer $f(x)=g(x)=x$ ise bu tür noktalara da $f$ ve $g$ ’nin ortak sabit noktaları denir.}

\definition [lipschitz fonksiyon] {$(X,d)$ bir metrik uzay ve $f:X\to X$ bir fonksiyon olsun. Bir $\alpha >0$ reel sayısı için $f$ fonksiyonu
 $$ d(f(x),f(y))\leq \alpha d(x,y) $$
 şartını sağlıyorsa bu fonksiyona lipschitz fonksiyonu, $\alpha$ reel sayısına da lipschitz sabiti denir.}

\definition [daraltan fonksiyon] {$(X,d)$ bir metrik uzay ve $f:X\to X$ bir fonksiyon olsun. Bir $0\leq \alpha <1$ reel sayısı için $f$ fonksiyonu 
$$d(f(x),f(y))\leq \alpha d(x,y) $$
 şartını sağlıyorsa bu fonksiyona daraltan fonksiyon denir. $\alpha = 1$ için bu şartı sağlayan $f$ fonksiyonuna genişlemeyen fonksiyon denir.}

\definition [kesin daraltan fonksiyon] {$(X,d)$ bir metrik uzay ve $f:X\to X$ bir fonksiyon olmak üzere $x\neq y$ için 
$$d(f(x),f(y))< d(x,y) $$
 şartını sağlıyorsa bu fonksiyona kesin daraltan fonksiyon denir.}
\begin{center}
 daraltan $\Longrightarrow$ kesin daraltan $\Longrightarrow$ genişlemeyen $\Longrightarrow$ Lipschitz
\end{center}

\definition [mesafe değiştiren fonksiyon] {$\psi : [0,+\infty)\to [0,+\infty)$ fonksiyonu
  \begin{itemize}
  \item $\psi$ sürekli ve azalmayan(monoton artan)dir.
  \item $\psi (t)=0 \iff t=0$
  \end{itemize}
şartlarıni sağlıyorsa bu fonksiyona mesafe değiştiren fonksiyon denir. Her mesafe değiştiren fonksiyon bir metriktir. Fakat tersi her zaman doğru değildir. Ters örnek: $\psi(t)=t^2$.}

\definition [Picard iterasyonu] {$(X,d)$ bir metik uzay ve $f : X\to X$ bir dönüşüm olsun. $x_0\in X$ ve $x_1=f(x_0)$, $x_2=f(x_1),\cdots$ olmak üzere 
$$x_{n+1}=f(x_n), \quad n=0,1,2,\cdots$$
şeklindeki ifadede $x_n$ 'e $x_0$ noktasının $f$ altındaki $n$. Picard iterasyonu olarak adlandırılır.}

\definition [asimptotik regulerlik] {$(X,d)$ bir metrik uzay ve $f:X\to X$ bir dönüşüm olmak üzere bir $x_0\in X$ noktası için 
$$\lim_{n\to \infty}d(f^n(x_0),f^{n+1}(x_0))=0 $$
oluyorsa $f$ dönüşümü $x_0$ noktasında asimptotik regulerdir denir.}

\definition [süreklilik] {$(X,d$ ve $(Y,\rho )$ iki metrik uzay, $f:X\to Y$ bir dönüşüm ve $x_ 0\in X$ olsun. Her $\varepsilon>0$ için $d(x,y)<\delta$ olduğunda $\rho (f(x),f(x_0))<\varepsilon$ veya buna denk ifadeyle $f(B(x_0 ;\delta ))\subseteq B(f(x_0);\varepsilon)$ olacak şekilde bir $\delta>0$ varsa, $f$ ye $x_0$ noktasında süreklidir denir. Eğer $f$ , $X$ in her noktasında sürekli ise, $f$ ye $X$ de süreklidir denir.}

\definition [alttan yarısürekli] {$(X,d)$ bir metrik uzay ve $f:X\to \mathbb{R}$ bir dönüşüm olsun. $x_0\in X$ olmak üzere her $x\in X$ için 
$$\liminf_{x\to x_0}f(x)\geq f(x_0)$$
oluyorsa $f$ dönüşümü $x_0$ noktasında alttan yarısüreklidir denir. Veya $X$'teki $x_0$'a yakınsayan  her $(x_n)$ dizisi için 
$$\lim_{n\to \infty}x_n=x_0\Rightarrow \liminf_{n\to \infty}f(x_n)\geq f(x_0)$$
oluyorsa $f$ dönüşümü $x_0$ noktasında alttan yarısüreklidir denir.}

\definition [üstten yarısürekli] {$(X,d)$ bir metrik uzay ve $f:X\to \mathbb{R}$ bir dönüşüm olsun. $x_0\in X$ olmak üzere her $x\in X$ için 
$$\limsup_{x\to x_0}f(x)\leq f(x_0)$$
oluyorsa $f$ dönüşümü $x_0$ noktasında üstten yarısüreklidir denir.Veya $X$'teki $x_0$'a yakınsayan her $(x_n)$ dizisi için 
$$\lim_{n\to \infty}x_n=x_0\Rightarrow \limsup_{n\to \infty}f(x_n)\leq f(x_0)$$
oluyorsa $f$ dönüşümü $x_0$ noktasında üstten yarısüreklidir denir.}

\definition [metrik uzay] {$X$ boş olmayan bir küme ve $d:X\times X \to \mathbb{R}$ bir fonksiyon olsun. Her $x,y,z\in X$ için

\begin{itemize}
\item $d(x,y)=0\iff x=y$
\item $d(x,y)=d(y,x)$
\item $d(x,y)\leq d(x,z)+d(z,y)$
\end{itemize}

şartları sağlanıyorsa $d$ fonksiyonuna $X$ üzerinde bir metrik ve $d$ ile birlikte $X$'e metrik uzay denir ve bu metrik uzay $(X,d)$ ile gosterilir.}

\definition [yarımetrik uzay] {$X$ boş olmayan bir küme ve $d:X\times X \to \mathbb{R}$ bir fonksiyon olsun. Her $x,y,z\in X$ için

\begin{itemize}
\item $d(x,y)=0\iff x=y$
\item $d(x,y)=d(y,x)$
\item $d(x,y)\leq d(x,z)+d(z,y)$
\end{itemize}

şartları sağlanıyorsa $d$ fonksiyonuna $X$ üzerinde bir yarımetrik ve $d$ ile birlikte $X$'e yarımetrik uzay denir ve bu metrik uzay $(X,d)$ ile gosterilir.}

\definition [quasi-metrik uzay] {$X$ boş olmayan bir küme ve $d:X\times X \to \mathbb{R}$ bir fonksiyon olsun. Her $x,y,z\in X$ için

\begin{itemize}
\item $d(x,y)=0\iff x=y$
\item $d(x,y)\leq d(x,z)+d(z,y)$
\end{itemize}

şartları sağlanıyorsa $d$ fonksiyonuna $X$ üzerinde bir quasi-metrik ve $d$ ile birlikte $X$'e quasi-metrik uzay denir ve bu metrik uzay $(X,d)$ ile gosterilir.}

\definition [cauchy dizisi] {$(X,d)$ bir metrik uzay ve $\{x_n\}$ de bu uzayda bir dizi olsun. Her $\varepsilon>0$ için $m,n>N$ olduğunda $d(x_n,x_m)<\varepsilon$ olacak bicimde $N(\varepsilon)\in \mathbb{N}$ sayısı varsa $\{x_n\}$ dizisine Cauchy dizisi denir.}

\definition [tam metrik uzay] {$(X,d)$ metrik uzayındaki her cauchy dizisi yine bu uzayda bir noktaya yakınsıyorsa bu uzaya tam metrik uzay denir.}

\definition [vektor(lineer) uzayı] {$V$ boş olmayan bir küme, $F$ bir cisim ve $+:V\times V\to V$ ve $\cdot:F\times V\to V$ işlemleri aşağıdaki şartları sağlasin:
  \begin{itemize}
  \item [V1.] Her $x,y\in V$ için $x+y\in V$ dir.
  \item [V2.] Her $x,y,z\in V$ için $x+(y+z)=(x+y)+z$ dir.
  \item [V3.] Her $x\in V$ için $x+\theta=\theta +x=x$ olacak bicimde $\theta\in V$ vardir. 
  \item [V4.] Her $x\in V$ için $x+(-x)=(-x)+x=\theta$ olacak bicimde $-x\in V$ vardir.
  \item [V5.] Her $x,y\in V$ için $x+y=y+x$ dir.
  \item [V6.] Her $x\in V$ ve her $\alpha \in F$ için $\alpha \cdot x\in V$ dir.
  \item [V7.] Her $x\in V$ ve her $\alpha, \beta \in F$ için $(\alpha \beta )\cdot x=\alpha(\beta \cdot x)$ dir.
  \item [V8.] Her $x\in V$ ve her $\alpha, \beta \in F$ için $(\alpha + \beta)\cdot x=\alpha \cdot x+ \beta \cdot x$ dir.
  \item [V9.] Her $x,y\in V$ ve her $\alpha \in F$ için $\alpha \cdot (x+y)=\alpha \cdot x+\alpha \cdot y$ dir.
  \item [V10.] Her $x \in V$ için $1\cdot x=x$ dir.
  \end{itemize}
Bu şartları sağlaniyorsa $V$'ye $F$ cismi üzerinde vektor uzayı denir. Ozel olarak $F=\mathbb{R}$ alınırsa reel vektör uzayı, $\mathbb{C}$ alınırsa kompleks vektör uzayı denir.}

\definition [normlu uzay] {$N$, $F$ cismi üzerinde bir vektör uzayı olsun. $\|\cdot \|:N\to \mathbb{R}$ bir fonksiyon ve bu fonksiyonun bir $x\in N$'deki değeri de $\| x\|$ ile gösterilsin. Her $x,y\in N$ için
  \begin{itemize}
  \item [N1] $\| x\|=0 \iff x=\theta $
  \item [N2] $\| \alpha \cdot x\|=|\alpha |\| x\|, \quad (\alpha \in F)$
  \item [N3] $\| x+y\|\leq \| x\|+\| y\|$
  \end{itemize}
şartları sağlanıyorsa bu $\| \cdot \|$ fonksiyonuna norm, bu fonksiyonla birlikte $N$ vektor uzayına normlu uzay denir.}

\definition [Banach uzay] {$N$ bir normlu uzay ve $d(x,y)=\| x-y\|$ şeklinde tanımlanan fonksiyon da gerçekten bir metriktir. Bu metriğe göre tam olan $N$ normlu uzayına Banach uzayı denir.}

\definition [iç çarpım uzayı] {$N$, $F$ cismi üzerinde bir vektör uzayı olsun. $\langle \cdot , \cdot \rangle :N\times N \to F$ de fonksiyonu her $x,y,z \in N$ için
  \begin{itemize}
  \item [İ1] $\langle x+y,z\rangle =\langle x,z\rangle +\langle y,z\rangle$
  \item [İ2] $\langle x,y \rangle =\langle y,x\rangle $
  \item [İ3] $\langle \alpha \cdot x,y\rangle =\alpha \langle x,y\rangle , \quad (\alpha \in F)$
  \item [İ4] $\langle x,x\rangle \geq 0$ ve $\langle x,x\rangle =0\iff x=\theta$
  \end{itemize}
şartlarını sağlıyorsa bu fonksiyona iç çarpım fonksiyonu denir. Bu fonksiyonla birlikte $N$ vektör uzayına iç çarpım uzayı veya ön-Hilbert uzayı denir. }

\definition [Hilbert Uzayı] {$H$ bir iç çarpım uzayı olmak üzere  $\| x \| =\sqrt{\langle x,x \rangle }$ bir norm ve $d(x,y)=\| x-y \| =\sqrt{\langle x-y, x-y \rangle }$ bir metrik tanımlar. Bu metriğe göre tam olan $H$ iç çarpım uzayına Hilbert uzayı denir.}

\definition [sınırlı küme] {$(X,d)$ bir metrik uzay ve $B \subset X$ olmak üzere her $x,y\in B$ için $d(x,y)\leq r$ olacak şekilde bir $r>0$ sayısı varsa $B$ kümesine sınırlı küme denir.}

\definition [diameter-çap] {$(X,d)$ bir metrik uzay ve $B \subset X$ olmak üzere her $x,y\in B$ için $\delta(B)=\sup_{x,y\in B} d(x,y)$ sayısına $B$ kümesinin çapı denir.}

\definition [totally bounded-tamamen sınırlı(precompact-önkompakt)] {$(X,d)$ bir metrik uzay ve $B \subset X$ olmak üzere her $\varepsilon >0$ için 
$$B\subseteq \bigcup_{n=1}^{N}B_\varepsilon(a_n),\quad \big(B_\varepsilon(a_n)=\{x: d(x,a_n)<\varepsilon \}\big)$$
olacak şekilde $a_1, a_2,\cdots ,a_N\in X$ sonlu sayıda nokta vardır. A uniformly continuous function maps totally bounded sets to totally bounded sets.A totally bounded set is geometrically ‘finite’, so an infinite sequence of points in a totally bounded set is caged in, so to speak, with nowhere to escape to:
A set $B$ is totally bounded $\iff$ Every sequence in $B$ has a Cauchy subsequence.}

\definition [kompakt küme] {$(X,d)$ bir metrik uzay ve $K\subseteq X$ olsun. $K$ kümesinin her açık örtüsünün yine $K$ kümesini örten onlu bir alt örtüsü varsa $K$ kümesine kompakt küme denir.}

\definition [convex-konveks] {$V$ bir vektör uzay ve $A\subseteq V$ olsun. Eğer $\lambda\in [0,1]$ olmak üzere her $u,v\in A$ için
$$\lambda u+(1-\lambda)v\in A $$
oluyorsa $A$ kümesine konveks küme denir.}

\definition [convex hull] {$V$ bir vektör uzay ve $A\subseteq V$ olmak üzere $A$ kümesini içeren tüm konveks kümelerin kesişimine (yani en küçük konveks kümeye) convex hull denir.}

\definition [b-Metrik Uzayı] {$X$ boş olmayan bir küme ve $b\geq 1$ bir reel sayı olsun. $d:X\times X\to \mathbb{R}^+$ fonksiyonu her $x,y,z\in X$ için aşağıdaki şartları sağlasın:
  \begin{itemize}
  \item $d(x,y)=0\iff x=y$
  \item $d(x,y)=d(y,x)$
  \item $d(x,z)\leq b[d(x,y)+d(y,z)]$
  \end{itemize}
Bu şekildeki $(X,d)$ metrik uzayıda b-metrik uzayı denir. $b=1$ için metrik uzay elde ederiz. $b$-metrik uzay, metrik uzayların bir genelleştirmesidir. Yakınsaklık, cauchy dizisi ve tamlık özellikleri metrik uzaylardaki şekliyle aynı şekilde tanımlanır. In a b-metric space (i) a convergent sequence has a unique limit, (ii) each convergent sequence is Cauchy, (iii) in general, a b-metric is not continuous.}




\section{TEOREMLER}

\theorem [banach daralma ilkesi] {$(X,d)$ bir tam metrik uzay ve $f:X\to X$ bir daraltan fonksiyon olsun. $f$ fonksiyonunun bu uzayda bir tek sabit noktası vardır. Dahası $x$ noktasının $f$ altındaki $n$-inci iterasyonu olan $f^n(x)$,  $f^0(x)=x, \quad f^{n+1}(x)=f(f^n(x))$ şeklinde tanımlansın. $\lim_{n\to \infty} f^n(x)$ bu fonksiyonun sabit noktasıdır. }

\theorem []

\section{\"ORNEKLER}
\example [sabit nokta] {$f,g:[0,1]\to [0,1]$, $f(x)=x^2$ fonksiyonu için $x=0$ ve $x=1$ noktaları sabit noktalardır. Gerçekten $f(0)=0$ ve $f(1)=1$ dir. Fakat $g(x)=x-1$ fonksiyonun hiçbir sabit noktası yoktur.}

\example [periyodik-sabit nokta] {$f:(0,1)\to ()0,1$, $f(x)=\frac{1}{x}$ fonksiyonunun sabit noktası yoktur. Fakat $f^2(x)=x$ olduğundan her $x\in (0,1)$ noktası $f$ nin periyodik noktasıdır. }

\example [daraltan dönüşüm] {$f:(0,1]\to (0,1]$, $f(x)=\frac{x}{3}$ fonksiyonu $$|f(x)-f(y)|=\frac{1}{3}|x-y|\leq\frac{1}{3}|x-y|$$ olup daraltan bir dönüşümdür. Fakat sabit noktası yoktur, bu $(0,1]$'in tam olmamasından kaynaklanır.}

\example [tamlık-kompaktlık] {$\mathbb{R}$ alışılmış metrikle tamdır fakat kompakt değildir.}

\example [sabit nokta-kesin daraltan] {$f:[1,+\infty)\to [1,+\infty)$ olmak üzere $f(x)=x+\frac{1}{x}$ şeklinde tanımlanan fonksiyon kesin daraltandır. O halde
      \begin{align*}
    |f(x)-f(y)|&=\big|x+\frac{1}{x}-y-\frac{1}{y}\big|\\
&=\big|x-y+\frac{1}{x}-\frac{1}{y}\big|\\
&=|x-y|\big|1-\frac{1}{xy}\big|\\
&<|x-y|
      \end{align*}
 elde edilir. Çünkü birbirinden farklı her $x,y\in [1,+\infty)$ için $0<1-\frac{1}{xy}<1$ dır.}

\example [sabit nokt-kesin daraltan] {$f:[0,1]\to [0,1]$, $f(x)=\frac{1}{2}x+\frac{1}{4}x^2$ fonksiyonu kesin daraltandır. Gerçekten
  \begin{align*}
    \big|\frac{1}{2}x+\frac{1}{4}x^2-\frac{1}{2}y-\frac{1}{4}y^2\big|&\leq \frac{1}{2}|x-y|+\frac{1}{4}|x+y||x-y|\\
&< \frac{1}{2}|x-y|+\frac{1}{2}|x-y|\\
&=|x-y|
  \end{align*}
olup daraltan olduğu görülebilir.

\example [daraltan dönüşüm] {$\mathbb{R}$ üzerinde tanımlı $f(x)=\cos x $ fonksiyonu açıkça görüleceği üzere daraltan dönüşüm değildir. Kabul edelim ki bir $h\in (0,1)$ olsun öyle ki her $x\neq y$ için
$$\big|\frac{\cos x-\cos y}{x-y}\big| \leq h$$
sağlansın. $y\to x$ iken her iki tarafın türevini alırsak $|sin x|\leq h$ olur her $x$ için. Fakat bu bir çelişkidir, çünkü $x=\frac{\pi}{2}$ için hiçbir $h\in (0,1)$ bulunamaz. 
Diğer yandan 2 
}

\end{document}