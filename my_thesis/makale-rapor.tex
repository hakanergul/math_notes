\documentclass[10pt]{amsart}
\usepackage[margin=0.5in]{geometry}
\usepackage{amsmath}
\usepackage{enumerate}
\usepackage{pxfonts}

\renewcommand{\abstractname}{\"OZET}
\theoremstyle{plain}
\newtheorem{theorem}{Teorem}
\newtheorem{proposition}{Onerme}
\newtheorem*{kanit}{\emph{Kan\i t}}

\theoremstyle{definition}
\newtheorem{definition}{Tan\i m}



\title{A fixed point theorem for weakly Zamfirescu mappings }
\author{Ariza-Ruiz, David Jimenez-Melado, Antonio Lopez-Acedo,Genaro}

\begin{document}

\maketitle

\begin{abstract}
\noindent Zamfirescu 1975'te Banach, Kannan ve Chatterjea tipi sabit nokta teoremlerini genelleyen bir sabit nokta teoremi yayinladi. Bu makalede Dugundji ve Granas'in yontemi kullanilarak bu teorem, zayif Zamfirescu donusumune genellenecek.
\end{abstract}

\section*{Giris}
Banach Daralma Ilkesi, Kannan ve Chatterjea tanimlari verilip bunlarin birbirinden bagimsiz olduklari belirtildi.

\definition $(X,d)$  metrik uzay ve $f: X \rightarrow X$ donusumu $\alpha \in [0,1)$ olmak uzere 
\begin{equation} 
  d(f(x), f(y))\leq \alpha\max\bigg\{ d(x,y), \frac{1}{2}[d(x,f(x))+d(y,f(y))], \frac{1}{2}[d(y,f(x))+d(y,f(x))] \bigg\} 
\tag*{[Z]}
\end{equation}
sartini sagliyorsa bu donusume \texttt{Zamfirescu donusumu} denir.

\begin{enumerate}[\textbf{Not} 1.]
\item  Tam metrik uzayda Zamfirescu donusumu bir tek sabit noktaya sahiptir.
\item  Zamfirescu donusumu, Banach Daralma Ilkesi, Kannan ve Chatterjea sabit nokta teoremlerinden daha geneldir.
\end{enumerate}

\subsection*{Zayif Daraltan Donusum}
  Dugundji ve Granas [7], Banach Daralma Ilkesindeki $\alpha \in [0,1)$ sayisi $\alpha(x,y)$ fonksiyonuyla degistirildi. Yani $f:X \rightarrow X$ donusumu,
 $\alpha :X\times X \rightarrow [0,1]$ fonksiyonu her $0<a\leq b$ icin 
\begin{equation*} 
  \sup\big\{\alpha(x,y) : 0\leq d(x,y) \leq b \big\} < 1
\end{equation*}
sartini saglayan bir fonksiyon olmak uzere her $x,y\in X$ icin
\begin{equation*} 
  d(f(x), f(y))\leq \alpha(x,y)d(x,y) 
\end{equation*}
oluyorsa $f$ donusumune \texttt{zayif daraltandir} denir.

Bu mantik kullanilarak [8] de Kannan donusumu, zayif Kannan donusumune genellestirildi. Chatterjea de benzer olarak denenebilir. Ciric'in sabit nokta teoremi bu yontemle zayif Ciric olamiyor. Ters ornek Sastry [9] 'da mevcut. Fakat Zamfirescu donusumu icin zayif Zamfirescu donusumu elde edebiliyoruz.

\subsection*{Zayif Zamfirescu Donusumleri}
 \definition  $(X,d)$ metrik uzay olsun. $f:X \rightarrow X$ donusumu,
 $\alpha :X\times X \rightarrow [0,1]$ fonksiyonu her $0<a\leq b$ icin 
\begin{equation*} 
  \sup\big\{\alpha(x,y) : 0\leq d(x,y) \leq b \big\} < 1
\end{equation*}
sartini saglayan bir fonksiyon ve
\begin{equation*} 
  M_f= \max\bigg\{ d(x,y), \frac{1}{2}[d(x,f(x))+d(y,f(y))], \frac{1}{2}[d(y,f(x))+d(y,f(x))] \bigg\}
\end{equation*}
olmak uzere her $x,y\in X$ icin
\begin{equation*} 
  d(f(x), f(y))\leq \alpha(x,y)M_f(x,y) 
\end{equation*}
oluyorsa $f$ donusumune \texttt{zayif Zamfirescu donusumu} denir.

\proposition  $(X,d)$ bir metrik uzay ve $D \subset X$ olsun. $f:D\rightarrow X$ bir zayif Zamfirescu donusumu ise $D$ de en fazla bir sabit noktasi vardir.
\kanit $u$ ve $v$, $f$ donusumunun birbirinden farkli iki sabit noktasi olsun.$0<d(u,v)=r$ alinirsa

\begin{equation*}
  \theta(r/2,r)= \sup\big\{\alpha(x,y) : \frac{r}{2}\leq d(x,y) \leq r \big\} < 1
\end{equation*}
olmak uzere  $\alpha(u,v)\leq \theta(r/2, r)<1$ yazilabilir. Buradan
\begin{align*}
d(u,v)&=d(f(u),f(v))\\ &\leq \alpha(u,v)d(u,v)\\
                    &\leq \theta(r/2,r)d(u,v)
\end{align*}
olup $d(u,v)$ leri sadelestirirsek $1<\theta(r/2,r)$ elde ederiz. Bu celiskidir. Demekki $u=v$ dir. \qed

\newpage
 Hatirlayalim ki $(X, d)$ metrik uzayinda tanimli $f: X\rightarrow X$ donusumu icin
  \begin{equation*}
    \lim_{n\rightarrow \infty}d(f^n(x_0),f^{n+1}(x_0))=0 
  \end{equation*}
oluyorsa $f$ donusumu $x_0\in X$ noktasinda asimptotik regulerdir denir.

\proposition $(X, d)$ bir metrik uzay ve $f : X\rightarrow X$ bir zayif Zamfiresu donusumu olsun. O zaman $f$ donusumu $X$ in her noktasinda asimptotik regulerdir.
\kanit Her halukarda $d(x_n,x_{n+1})\leq \alpha(x_{n-1},x_n)d(x_{n-1},x_n)$ saglanir. Buradan $\{d(x_n,x_{n+1})\}$ dizisi artmayan(nonincreasing) dizidir. O halde bir $r\geq 0$ sayisina yakinsar. $r=0$ oldugunu gostermek yeterlidir, bunun icin $r>0$ oldugunu kabul edip celiski elde edelim.

$\{d(x_n,x_{n+1})\}$ dizisi artmayan oldugu icin her $n\in \mathbb{N}$ icin $0<r\leq d(x_n,x_{n+1})\leq d(x_0,x_1)$ 'dir. Ve $\theta(r,d(x_0,x_1))$ icin


\end{document}