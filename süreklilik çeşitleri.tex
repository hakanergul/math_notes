
\documentclass[12pt]{article}
%%%%%%%%%%%%%%%%%%%%%%%%%%%%%%%%%%%%%%%%%%%%%%%%%%%%%%%%%%%%%%%%%%%%%%%%%%%%%%%%%%%%%%%%%%%%%%%%%%%%%%%%%%%%%%%%%%%%%%%%%%%%%%%%%%%%%%%%%%%%%%%%%%%%%%%%%%%%%%%%%%%%%%%%%%%%%%%%%%%%%%%%%%%%%%%%%%%%%%%%%%%%%%%%%%%%%%%%%%%%%%%%%%%%%%%%%%%%%%%%%%%%%%%%%%%%
\usepackage{amssymb}
\usepackage{amsfonts}
\usepackage{mitpress}

%TCIDATA{OutputFilter=LATEX.DLL}
%TCIDATA{Version=5.50.0.2953}
%TCIDATA{<META NAME="SaveForMode" CONTENT="1">}
%TCIDATA{BibliographyScheme=Manual}
%TCIDATA{Created=Tuesday, March 31, 2015 14:05:14}
%TCIDATA{LastRevised=Wednesday, April 01, 2015 02:55:54}
%TCIDATA{SuppressPackageManagement}
%TCIDATA{<META NAME="GraphicsSave" CONTENT="32">}
%TCIDATA{<META NAME="DocumentShell" CONTENT="Articles\SW\A Simple MIT Press Article">}
%TCIDATA{CSTFile=LaTeX Article (bright).cst}
%TCIDATA{PageSetup=72,72,72,72,0}
%TCIDATA{Counters=arabic,1}
%TCIDATA{AllPages=
%H=36
%F=36
%}


\newtheorem{theorem}{Theorem}
\newtheorem{acknowledgement}[theorem]{Acknowledgement}
\newtheorem{algorithm}[theorem]{Algorithm}
\newtheorem{axiom}[theorem]{Axiom}
\newtheorem{case}[theorem]{Case}
\newtheorem{claim}[theorem]{Claim}
\newtheorem{conclusion}[theorem]{Conclusion}
\newtheorem{condition}[theorem]{Condition}
\newtheorem{conjecture}[theorem]{Conjecture}
\newtheorem{corollary}[theorem]{Corollary}
\newtheorem{criterion}[theorem]{Criterion}
\newtheorem{definition}[theorem]{Tan\i m}
\newtheorem{example}[theorem]{Example}
\newtheorem{exercise}[theorem]{Exercise}
\newtheorem{lemma}[theorem]{Lemma}
\newtheorem{notation}[theorem]{Notation}
\newtheorem{problem}[theorem]{Problem}
\newtheorem{proposition}[theorem]{Proposition}
\newtheorem{remark}[theorem]{Not}
\newtheorem{solution}[theorem]{Solution}
\newtheorem{summary}[theorem]{Summary}
\newenvironment{proof}[1][Proof]{\noindent\textbf{#1.} }{\ \rule{0.5em}{0.5em}}
\newdimen\dummy
\dummy=\oddsidemargin
\addtolength{\dummy}{72pt}
\marginparwidth=.5\dummy
\marginparsep=.1\dummy
\input{tcilatex}
\begin{document}

\title{S\"{u}reklilik \c{C}e\c{s}itleri}
\author{Hakan ERG\"{U}L}

\section{S\"{u}reklilik \c{C}e\c{s}itleri}

\begin{definition}
$A\subseteq 
%TCIMACRO{\U{211d} }%
%BeginExpansion
\mathbb{R}
%EndExpansion
$ bo\c{s} olmayan bir k\"{u}me, $f:A\rightarrow 
%TCIMACRO{\U{211d} }%
%BeginExpansion
\mathbb{R}
%EndExpansion
$ bir fonksiyon ve $a\in A$ olsun. Her $\varepsilon >0$ ve her $x\in A$ i%
\c{c}in%
\[
\left\vert x-a\right\vert <\delta \Rightarrow \left\vert
f(x)-f(a)\right\vert <\varepsilon 
\]%
\c{s}art\i n\i\ sa\u{g}layan $\delta =\delta (\varepsilon ,a)$ pozitif say\i
s\i\ varsa \textquotedblleft $f$ fonksiyonu $a$ noktas\i nda s\"{u}%
reklidir\textquotedblright\ denir.\qed
\end{definition}

\begin{definition}
$A\subseteq 
%TCIMACRO{\U{211d} }%
%BeginExpansion
\mathbb{R}
%EndExpansion
$ bo\c{s} olmayan bir k\"{u}me, $f:A\rightarrow 
%TCIMACRO{\U{211d} }%
%BeginExpansion
\mathbb{R}
%EndExpansion
$ bir fonksiyon ve $a\in A$ olsun. $D(a;\delta )$, $a$ n\i n $\delta $
civar\i\ olmak \"{u}zere her $\varepsilon >0$ i\c{c}in 
\[
f(A\cap D(a;\delta ))\subseteq D(f(a);\varepsilon )
\]%
olacak \c{s}ekilde $\delta =\delta (\varepsilon ,a)>0$ say\i s\i\ varsa $f$
fonksiyonu $a$ noktas\i nda s\"{u}reklidir.$\qed$
\end{definition}

\begin{definition}
$A\subseteq 
%TCIMACRO{\U{211d} }%
%BeginExpansion
\mathbb{R}
%EndExpansion
$ bo\c{s} olmayan bir k\"{u}me, $f:A\rightarrow 
%TCIMACRO{\U{211d} }%
%BeginExpansion
\mathbb{R}
%EndExpansion
$ bir fonksiyon ve $a\in A$ olsun.

\begin{enumerate}
\item E\u{g}er $a$\textbf{\ noktas\i\ }$A$\textbf{\ n\i n bir y\i \u{g}\i
lma noktas\i ysa} $f$ fonksiyonunun $a$ noktas\i nda s\"{u}rekli olmas\i\ i%
\c{c}in gerek ve yeter \c{s}art 
\[
\lim_{x\rightarrow a}f(x)=f(a)
\]%
olmas\i d\i r. Yani \c{s}u \"{u}\c{c} \c{s}art sa\u{g}lanmal\i d\i r:

\begin{enumerate}
\item $f$ fonksiyonu $a$ noktas\i nda tan\i ml\i\ olmal\i\ ($f(a)$ n\i n
anlaml\i\ olabilmesi i\c{c}in)

\item $f$ fonksiyonunun $a$ noktas\i nda limiti olmal\i 

\item $f$ fonksiyonunun $a$ noktas\i ndaki limiti, $f(a)$ ya e\c{s}it
olmal\i 
\end{enumerate}

\item E\u{g}er $a$\textbf{\ noktas\i\ }$A$\textbf{\ n\i n bir y\i \u{g}\i
lma noktas\i\ de\u{g}ilse,} yani \textbf{ayr\i k nokta} ise $f$ fonksiyonu $a
$ noktas\i nda a\c{s}ikar bi\c{c}imde s\"{u}rekli olur. $a$ n\i n ayr\i k
nokta olmas\i ndan dolay\i\ 
\[
\exists \delta _{0}>0\text{ \"{o}yle ki }A\cap D(a;\delta _{0})=\emptyset 
\]%
diyebiliriz. Tan\i m 2'den her $\varepsilon >0$ i\c{c}in%
\[
f(A\cap D(a;\delta _{0}))=f(\emptyset )=\emptyset \subseteq
D(f(a);\varepsilon )
\]%
olacak \c{s}ekilde $\delta _{0}>0$ say\i s\i\ bulduk. \.{I}spat\i m\i z
tamamlanm\i \c{s}t\i r.\qed
\end{enumerate}
\end{definition}

\begin{definition}
(\textbf{Par\c{c}al\i\ S\"{u}reklilik) }Bir fonksiyonun tan\i ml\i\ oldu\u{g}%
u aral\i k, fonksiyonun s\"{u}rekli oldu\u{g}u sonlu say\i da a\c{c}\i k(u%
\c{c} noktalar\i\ \c{c}\i kar\i lm\i \c{s}) alt aral\i \u{g}a b\"{o}l\"{u}%
nebiliyorsa ve fonksiyon her bir alt aral\i \u{g}\i n u\c{c} noktalar\i nda
sonlu bir limite sahipse, fonksiyon bu aral\i kta par\c{c}al\i\ s\"{u}%
reklidir denir. \qed
\end{definition}

\begin{definition}
\bigskip \textbf{(Dizisel S\"{u}reklilik) } $X$ ve $Y$ bo\c{s} olmayan k\"{u}%
meler, $f:X\rightarrow Y$ bir fonksiyon ve $a\in X$ olsun. $X$ k\"{u}%
mesindeki $a$ ya yak\i nsayan her $\left\{ x_{n}\right\} $ dizisi i\c{c}in $%
\left\{ f(x_{n})\right\} $ dizisi de $f(a)$ ya yak\i ns\i yorsa $f$
fonksiyonu $a$ noktas\i nda dizisel s\"{u}reklidir.$\qed$
\end{definition}

\begin{remark}
S\"{u}rekli her fonksiyon dizisel s\"{u}reklidir. Tersi her zaman do\u{g}ru
de\u{g}ildir. Ancak metrik uzaylarda dizisel s\"{u}reklilik, s\"{u}reklili%
\u{g}e denk olur.
\end{remark}

\begin{definition}
(\textbf{D\"{u}zg\"{u}n S\"{u}reklilik) }$A\subseteq 
%TCIMACRO{\U{211d} }%
%BeginExpansion
\mathbb{R}
%EndExpansion
$ bo\c{s} olmayan bir k\"{u}me ve $f:A\rightarrow 
%TCIMACRO{\U{211d} }%
%BeginExpansion
\mathbb{R}
%EndExpansion
$ bir fonksiyon olsun. Her $\varepsilon >0$ ve her $x,y\in A$ i\c{c}in%
\[
\left\vert x-y\right\vert <\delta \Rightarrow \left\vert
f(x)-f(y)\right\vert <\varepsilon 
\]%
olacak \c{s}ekilde yaln\i z $\varepsilon $ say\i s\i na ba\u{g}l\i\ olarak de%
\u{g}i\c{s}en bir $\delta =\delta (\varepsilon )$ pozitif say\i s\i\ varsa $f
$ fonksiyonu $A$ k\"{u}mesinde d\"{u}zg\"{u}n s\"{u}reklidir.\qed
\end{definition}

\begin{remark}
D\"{u}zg\"{u}n s\"{u}rekli her fonksiyon s\"{u}reklidir, fakat tersi her
zaman do\u{g}ru de\u{g}ildir. Ancak kapal\i\ ve s\i n\i rl\i\ bir aral\i kta
s\"{u}rekli fonksiyon, ayn\i\ zamanda d\"{u}zg\"{u}n s\"{u}reklidir.
\end{remark}

\begin{definition}
\textbf{(Mutlak S\"{u}reklilik) }$f:\left[ a,b\right] \rightarrow 
%TCIMACRO{\U{211d} }%
%BeginExpansion
\mathbb{R}
%EndExpansion
$ fonksiyonu verilsin. Her $\varepsilon >0$ ve her $\left\{
(a_{k},b_{k})\right\} _{k=1}^{n}$ $\subseteq (a,b)$ sonlu ayr\i k a\c{c}\i k
aral\i klar dizisi i\c{c}in%
\[
\dsum_{k=1}^{n}\left[ b_{k}-a_{k}\right] <\delta \Rightarrow
\dsum_{k=1}^{n}\left\vert f(b_{k})-f(a_{k})\right\vert <\varepsilon 
\]%
\c{s}art\i n\i\ sa\u{g}layan $\delta >0$ say\i s\i\ varsa \textquotedblleft $%
f$ fonksiyonu $\left[ a,b\right] $ aral\i \u{g}\i nda mutlak s\"{u}%
reklidir\textquotedblright\ denir.\qed
\end{definition}

\bigskip

\end{document}
